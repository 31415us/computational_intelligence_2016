
\section{Introduction}

\subsection{Sentiment Analysis}

In sentiment analysis we want to extract subjective information from text. It is
usually formulated as a text classification task where we want to distinguish
text sections by the attitude of the speaker or author. A sentence like
\textit{Just got my midterm and I'm impressed!} might be categorized as `positive'
whereas \textit{Exams are coming soon.} might be considered `negative'.

For our application we try to classify tweets into `positive' and `negative'
categories based on whether the contain a happy (`:)') or sad (`:(') smiley.

Real time sentiment analysis of social media posts in general can be particularly
useful for public relations and marketing purposes as it can provide insight into
customer satisfaction and related metrics.

\subsection{Baselines}

Most modern state of the art text classification systems are based on convolutional
neural networks. We will use a simple convolutional neural network similar to
the one presented in~\cite{cnn14} as a baseline for comparison.

As a second baseline we use a traditional feature based classifier. We extract
tfidf (term frequency-inverse document frequency) features from the tweets
and train a tree ensemble classifier using gradient boosting~\cite{gradBoost}.

\subsection{Contribution}

For our main contribution we extract additional features based on a clustering
of the vector space embedding of the words in a tweet. We first compute global
vectors~\cite{glove} for the words in our dataset and use simple k-means to cluster them.  
Additionally we will also train a LDA model~\cite{LDA} and use the resulting
topic mixture vectors as features.
